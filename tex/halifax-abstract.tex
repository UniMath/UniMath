\documentclass{amsart}

\title{Univalent FOLDS}
\begin{document}
\maketitle

The ``equivalence principle'' of higher category theory says that ``meaningful'' statements should be invariant under equivalence.
First-Order Logic with Dependent Sorts (FOLDS) was introduced by Makkai as a language for describing higher-categorical structures in which this would always be true, because there is no ``equality'' that can distinguish equivalent structures.
More recently, Homotopy Type Theory (HoTT) is a foundation for mathematics, in which Voevodsky's Univalence Axiom (UA) enforces the equivalence principle for $\infty$-groupoids by essentially \emph{defining} ``equal'' to mean ``equivalent''.

In previous work, by ``relativizing'' UA, we defined a notion of ``univalent'' or ``saturated'' (1-)category in HoTT that satisfies the principle of equivalence.
We now extend this to other higher-categorical structures by defining them a la FOLDS inside HoTT.
Any FOLDS-signature comes with a canonical notion of ``univalence'' for its structures, and such ``univalent structures'' satisfy the principle of equivalence.
Examples include $n$-categories, $\dagger$-categories, and ``doubly weak'' double categories.


\end{document}