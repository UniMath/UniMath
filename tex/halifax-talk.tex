\documentclass{beamer}
\usepackage{amssymb,amsmath,stmaryrd,mathrsfs,mathpartir}

%% Set this to true before loading if we're using the TAC style file.
%% Note that eventually, TAC requires everything to be in one source file.
\def\definetac{\newif\iftac}    % Can't define a \newif inside another \if!
\ifx\tactrue\undefined
  \definetac
  %% Guess whether we're using TAC by whether \state is defined.
  \ifx\state\undefined\tacfalse\else\tactrue\fi
\fi

% Similarly detect beamer
\def\definebeamer{\newif\ifbeamer}
\ifx\beamertrue\undefined
  \definebeamer
  %% Guess whether we're using Beamer by whether \uncover is defined.
  \ifx\uncover\undefined\beamerfalse\else\beamertrue\fi
\fi

\iftac\else\usepackage{amsthm}\fi
\usepackage[all,2cell]{xy}
%\UseAllTwocells
%\usepackage{tikz}
%\usetikzlibrary{arrows}
\ifbeamer\else
  \usepackage{enumitem}
  \usepackage{xcolor}
  \definecolor{darkgreen}{rgb}{0,0.45,0} 
  \usepackage[pagebackref,colorlinks,citecolor=darkgreen,linkcolor=darkgreen]{hyperref}
  \usepackage{cleveref,aliascnt}
\fi
\usepackage{mathtools}          % for all sorts of things
\usepackage{graphics}           % for \scalebox, used in \widecheck
\usepackage{ifmtarg}            % used in \jd
\usepackage{microtype}
%\usepackage{color,epsfig}
%\usepackage{fullpage}
%\usepackage{eucal}
%\usepackage{wasysym}
%\usepackage{txfonts}            % for \invamp, or for the nice fonts
\usepackage{braket}             % for \Set, etc.
\let\setof\Set
\usepackage{url}                % for citations to web sites
\usepackage{xspace}             % put spaces after a \command in text
%\usepackage{cite}               % compress and sort grouped citations (only use with numbered citations)

%%% Set the fonts
\usepackage{mathpazo}
\usepackage[scaled=0.95]{helvet}
\usepackage{courier}
\linespread{1.05} % Palatino looks better with this

\makeatletter
\let\ea\expandafter

%% Defining commands that are always in math mode.
\def\mdef#1#2{\ea\ea\ea\gdef\ea\ea\noexpand#1\ea{\ea\ensuremath\ea{#2}\xspace}}
\def\alwaysmath#1{\ea\ea\ea\global\ea\ea\ea\let\ea\ea\csname your@#1\endcsname\csname #1\endcsname
  \ea\def\csname #1\endcsname{\ensuremath{\csname your@#1\endcsname}\xspace}}

%% WIDECHECK
\DeclareRobustCommand\widecheck[1]{{\mathpalette\@widecheck{#1}}}
\def\@widecheck#1#2{%
    \setbox\z@\hbox{\m@th$#1#2$}%
    \setbox\tw@\hbox{\m@th$#1%
       \widehat{%
          \vrule\@width\z@\@height\ht\z@
          \vrule\@height\z@\@width\wd\z@}$}%
    \dp\tw@-\ht\z@
    \@tempdima\ht\z@ \advance\@tempdima2\ht\tw@ \divide\@tempdima\thr@@
    \setbox\tw@\hbox{%
       \raise\@tempdima\hbox{\scalebox{1}[-1]{\lower\@tempdima\box
\tw@}}}%
    {\ooalign{\box\tw@ \cr \box\z@}}}

%% SIMPLE COMMANDS FOR FONTS AND DECORATIONS

\newcount\foreachcount

\def\foreachletter#1#2#3{\foreachcount=#1
  \ea\loop\ea\ea\ea#3\@alph\foreachcount
  \advance\foreachcount by 1
  \ifnum\foreachcount<#2\repeat}

\def\foreachLetter#1#2#3{\foreachcount=#1
  \ea\loop\ea\ea\ea#3\@Alph\foreachcount
  \advance\foreachcount by 1
  \ifnum\foreachcount<#2\repeat}

% Script: \sA is \mathscr{A}
\def\definescr#1{\ea\gdef\csname s#1\endcsname{\ensuremath{\mathscr{#1}}\xspace}}
\foreachLetter{1}{27}{\definescr}
% Calligraphic: \cA is \mathcal{A}
\def\definecal#1{\ea\gdef\csname c#1\endcsname{\ensuremath{\mathcal{#1}}\xspace}}
\foreachLetter{1}{27}{\definecal}
% Bold: \bA is \mathbf{A}
\def\definebold#1{\ea\gdef\csname b#1\endcsname{\ensuremath{\mathbf{#1}}\xspace}}
\foreachLetter{1}{27}{\definebold}
% Blackboard Bold: \lA is \mathbb{A}
\def\definebb#1{\ea\gdef\csname l#1\endcsname{\ensuremath{\mathbb{#1}}\xspace}}
\foreachLetter{1}{27}{\definebb}
% Fraktur: \ka is \mathfrak{a} (except when it's \kappa, see below), \kA is \mathfrak{A}
\def\definefrak#1{\ea\gdef\csname k#1\endcsname{\ensuremath{\mathfrak{#1}}\xspace}}
\foreachletter{1}{27}{\definefrak}
\foreachLetter{1}{27}{\definefrak}
% Sans serif
\def\definesf#1{\ea\gdef\csname i#1\endcsname{\ensuremath{\mathsf{#1}}\xspace}}
\foreachletter{1}{6}{\definesf}
\foreachletter{7}{14}{\definesf}
\foreachletter{15}{27}{\definesf}
\foreachLetter{1}{27}{\definesf}
% Bar: \Abar is \overline{A}, \abar is \overline{a}
\def\definebar#1{\ea\gdef\csname #1bar\endcsname{\ensuremath{\overline{#1}}\xspace}}
\foreachLetter{1}{27}{\definebar}
\foreachletter{1}{8}{\definebar} % \hbar is something else!
\foreachletter{9}{15}{\definebar} % \obar is something else!
\foreachletter{16}{27}{\definebar}
% Tilde: \Atil is \widetilde{A}, \atil is \widetilde{a}
\def\definetil#1{\ea\gdef\csname #1til\endcsname{\ensuremath{\widetilde{#1}}\xspace}}
\foreachLetter{1}{27}{\definetil}
\foreachletter{1}{27}{\definetil}
% Hats: \Ahat is \widehat{A}, \ahat is \widehat{a}
\def\definehat#1{\ea\gdef\csname #1hat\endcsname{\ensuremath{\widehat{#1}}\xspace}}
\foreachLetter{1}{27}{\definehat}
\foreachletter{1}{27}{\definehat}
% Checks: \Achk is \widecheck{A}, \achk is \widecheck{a}
\def\definechk#1{\ea\gdef\csname #1chk\endcsname{\ensuremath{\widecheck{#1}}\xspace}}
\foreachLetter{1}{27}{\definechk}
\foreachletter{1}{27}{\definechk}
% Underline: \uA is \underline{A}, \ua is \underline{a}
\def\defineul#1{\ea\gdef\csname u#1\endcsname{\ensuremath{\underline{#1}}\xspace}}
\foreachLetter{1}{27}{\defineul}
\foreachletter{1}{27}{\defineul}

% Particular commands for typefaces, sometimes with the first letter
% different.
\def\autofmt@n#1\autofmt@end{\mathrm{#1}}
\def\autofmt@b#1\autofmt@end{\mathbf{#1}}
\def\autofmt@l#1#2\autofmt@end{\mathbb{#1}\mathsf{#2}}
\def\autofmt@c#1#2\autofmt@end{\mathcal{#1}\mathit{#2}}
\def\autofmt@s#1#2\autofmt@end{\mathscr{#1}\mathit{#2}}
\def\autofmt@f#1\autofmt@end{\mathsf{#1}}
\def\autofmt@k#1\autofmt@end{\mathfrak{#1}}
% Particular commands for decorations.
\def\autofmt@u#1\autofmt@end{\underline{\smash{\mathsf{#1}}}}
\def\autofmt@U#1\autofmt@end{\underline{\underline{\smash{\mathsf{#1}}}}}
\def\autofmt@h#1\autofmt@end{\widehat{#1}}
\def\autofmt@r#1\autofmt@end{\overline{#1}}
\def\autofmt@t#1\autofmt@end{\widetilde{#1}}
\def\autofmt@k#1\autofmt@end{\check{#1}}

% Defining multi-letter commands.  Use this like so:
% \autodefs{\bSet\cCat\cCAT\kBicat\lProf}
\def\auto@drop#1{}
\def\autodef#1{\ea\ea\ea\@autodef\ea\ea\ea#1\ea\auto@drop\string#1\autodef@end}
\def\@autodef#1#2#3\autodef@end{%
  \ea\def\ea#1\ea{\ea\ensuremath\ea{\csname autofmt@#2\endcsname#3\autofmt@end}\xspace}}
\def\autodefs@end{blarg!}
\def\autodefs#1{\@autodefs#1\autodefs@end}
\def\@autodefs#1{\ifx#1\autodefs@end%
  \def\autodefs@next{}%
  \else%
  \def\autodefs@next{\autodef#1\@autodefs}%
  \fi\autodefs@next}

%% FONTS AND DECORATION FOR GREEK LETTERS

%% the package `mathbbol' gives us blackboard bold greek and numbers,
%% but it does it by redefining \mathbb to use a different font, so that
%% all the other \mathbb letters look different too.  Here we import the
%% font with bb greek and numbers, but assign it a different name,
%% \mathbbb, so as not to replace the usual one.
\DeclareSymbolFont{bbold}{U}{bbold}{m}{n}
\DeclareSymbolFontAlphabet{\mathbbb}{bbold}
\newcommand{\lDelta}{\ensuremath{\mathbbb{\Delta}}\xspace}
\newcommand{\lone}{\ensuremath{\mathbbb{1}}\xspace}
\newcommand{\ltwo}{\ensuremath{\mathbbb{2}}\xspace}
\newcommand{\lthree}{\ensuremath{\mathbbb{3}}\xspace}

% greek with bars
\newcommand{\albar}{\ensuremath{\overline{\alpha}}\xspace}
\newcommand{\bebar}{\ensuremath{\overline{\beta}}\xspace}
\newcommand{\gmbar}{\ensuremath{\overline{\gamma}}\xspace}
\newcommand{\debar}{\ensuremath{\overline{\delta}}\xspace}
\newcommand{\phibar}{\ensuremath{\overline{\varphi}}\xspace}
\newcommand{\psibar}{\ensuremath{\overline{\psi}}\xspace}
\newcommand{\xibar}{\ensuremath{\overline{\xi}}\xspace}
\newcommand{\ombar}{\ensuremath{\overline{\omega}}\xspace}

% greek with tildes
\newcommand{\altil}{\ensuremath{\widetilde{\alpha}}\xspace}
\newcommand{\betil}{\ensuremath{\widetilde{\beta}}\xspace}
\newcommand{\gmtil}{\ensuremath{\widetilde{\gamma}}\xspace}
\newcommand{\phitil}{\ensuremath{\widetilde{\varphi}}\xspace}
\newcommand{\psitil}{\ensuremath{\widetilde{\psi}}\xspace}
\newcommand{\xitil}{\ensuremath{\widetilde{\xi}}\xspace}
\newcommand{\omtil}{\ensuremath{\widetilde{\omega}}\xspace}

% MISCELLANEOUS SYMBOLS
\let\del\partial
\mdef\delbar{\overline{\partial}}
\let\sm\wedge
\newcommand{\dd}[1]{\ensuremath{\frac{\partial}{\partial {#1}}}}
\newcommand{\inv}{^{-1}}
\newcommand{\dual}{^{\vee}}
\mdef\hf{\textstyle\frac12 }
\mdef\thrd{\textstyle\frac13 }
\mdef\qtr{\textstyle\frac14 }
\let\meet\wedge
\let\join\vee
\let\dn\downarrow
\newcommand{\op}{^{\mathrm{op}}}
\newcommand{\co}{^{\mathrm{co}}}
\newcommand{\coop}{^{\mathrm{coop}}}
\let\adj\dashv
\SelectTips{cm}{}
\newdir{ >}{{}*!/-10pt/@{>}}    % extra spacing for tail arrows in XYpic
\newcommand{\pushout}[1][dr]{\save*!/#1+1.2pc/#1:(1,-1)@^{|-}\restore}
\newcommand{\pullback}[1][dr]{\save*!/#1-1.2pc/#1:(-1,1)@^{|-}\restore}
\let\iso\cong
\let\eqv\simeq
\let\cng\equiv
\mdef\Id{\mathrm{Id}}
\mdef\id{\mathrm{id}}
\alwaysmath{ell}
\alwaysmath{infty}
\alwaysmath{odot}
\def\frc#1/#2.{\frac{#1}{#2}}   % \frc x^2+1 / x^2-1 .
\mdef\ten{\mathrel{\otimes}}
\let\bigten\bigotimes
\mdef\sqten{\mathrel{\boxtimes}}
\def\lt{<}                      % For iTex compatibility
\def\gt{>}

%% OPERATORS
\DeclareMathOperator\lan{Lan}
\DeclareMathOperator\ran{Ran}
\DeclareMathOperator\colim{colim}
\DeclareMathOperator\coeq{coeq}
\DeclareMathOperator\eq{eq}
\DeclareMathOperator\Tot{Tot}
\DeclareMathOperator\cosk{cosk}
\DeclareMathOperator\sk{sk}
%\DeclareMathOperator\im{im}
\DeclareMathOperator\Spec{Spec}
\DeclareMathOperator\Ho{Ho}
\DeclareMathOperator\Aut{Aut}
\DeclareMathOperator\End{End}
\DeclareMathOperator\Hom{Hom}
\DeclareMathOperator\Map{Map}

%% ARROWS
% \to already exists
\newcommand{\too}[1][]{\ensuremath{\overset{#1}{\longrightarrow}}}
\newcommand{\ot}{\ensuremath{\leftarrow}}
\newcommand{\oot}[1][]{\ensuremath{\overset{#1}{\longleftarrow}}}
\let\toot\rightleftarrows
\let\otto\leftrightarrows
\let\Impl\Rightarrow
\let\imp\Rightarrow
\let\toto\rightrightarrows
\let\into\hookrightarrow
\let\xinto\xhookrightarrow
\mdef\we{\overset{\sim}{\longrightarrow}}
\mdef\leftwe{\overset{\sim}{\longleftarrow}}
\let\mono\rightarrowtail
\let\leftmono\leftarrowtail
\let\cof\rightarrowtail
\let\leftcof\leftarrowtail
\let\epi\twoheadrightarrow
\let\leftepi\twoheadleftarrow
\let\fib\twoheadrightarrow
\let\leftfib\twoheadleftarrow
\let\cohto\rightsquigarrow
\let\maps\colon
\newcommand{\spam}{\,:\!}       % \maps for left arrows
\def\acof{\mathrel{\mathrlap{\hspace{3pt}\raisebox{4pt}{$\scriptscriptstyle\sim$}}\mathord{\rightarrowtail}}}

% diagxy redefines \to, along with \toleft, \two, \epi, and \mon.

%% EXTENSIBLE ARROWS
\let\xto\xrightarrow
\let\xot\xleftarrow
% See Voss' Mathmode.tex for instructions on how to create new
% extensible arrows.
\def\rightarrowtailfill@{\arrowfill@{\Yright\joinrel\relbar}\relbar\rightarrow}
\newcommand\xrightarrowtail[2][]{\ext@arrow 0055{\rightarrowtailfill@}{#1}{#2}}
\let\xmono\xrightarrowtail
\let\xcof\xrightarrowtail
\def\twoheadrightarrowfill@{\arrowfill@{\relbar\joinrel\relbar}\relbar\twoheadrightarrow}
\newcommand\xtwoheadrightarrow[2][]{\ext@arrow 0055{\twoheadrightarrowfill@}{#1}{#2}}
\let\xepi\xtwoheadrightarrow
\let\xfib\xtwoheadrightarrow
% Let's leave the left-going ones until I need them.

%% EXTENSIBLE SLASHED ARROWS
% Making extensible slashed arrows, by modifying the underlying AMS code.
% Arguments are:
% 1 = arrowhead on the left (\relbar or \Relbar if none)
% 2 = fill character (usually \relbar or \Relbar)
% 3 = slash character (such as \mapstochar or \Mapstochar)
% 4 = arrowhead on the left (\relbar or \Relbar if none)
% 5 = display mode (\displaystyle etc)
\def\slashedarrowfill@#1#2#3#4#5{%
  $\m@th\thickmuskip0mu\medmuskip\thickmuskip\thinmuskip\thickmuskip
   \relax#5#1\mkern-7mu%
   \cleaders\hbox{$#5\mkern-2mu#2\mkern-2mu$}\hfill
   \mathclap{#3}\mathclap{#2}%
   \cleaders\hbox{$#5\mkern-2mu#2\mkern-2mu$}\hfill
   \mkern-7mu#4$%
}
% Here's the idea: \<slashed>arrowfill@ should be a box containing
% some stretchable space that is the "middle of the arrow".  This
% space is created as a "leader" using \cleader<thing>\hfill, which
% fills an \hfill of space with copies of <thing>.  Here instead of
% just one \cleader, we use two, with the slash in between (and an
% extra copy of the filler, to avoid extra space around the slash).
\def\rightslashedarrowfill@{%
  \slashedarrowfill@\relbar\relbar\mapstochar\rightarrow}
\newcommand\xslashedrightarrow[2][]{%
  \ext@arrow 0055{\rightslashedarrowfill@}{#1}{#2}}
\mdef\hto{\xslashedrightarrow{}}
\mdef\htoo{\xslashedrightarrow{\quad}}
\let\xhto\xslashedrightarrow

%% To get a slashed arrow in XYmatrix, do
% \[\xymatrix{A \ar[r]|-@{|} & B}\]
%% To get it in diagxy, do
% \morphism/{@{>}|-*@{|}}/[A`B;p]

%% Here is an \hto for diagxy:
% \def\htopppp/#1/<#2>^#3_#4{\:%
% \ifnum#2=0%
%    \setwdth{#3}{#4}\deltax=\wdth \divide \deltax by \ul%
%    \advance \deltax by \defaultmargin  \ratchet{\deltax}{100}%
% \else \deltax #2%
% \fi%
% \xy\ar@{#1}|-@{|}^{#3}_{#4}(\deltax,0) \endxy%
% \:}%
% \def\htoppp/#1/<#2>^#3{\ifnextchar_{\htopppp/#1/<#2>^{#3}}{\htopppp/#1/<#2>^{#3}_{}}}%
% \def\htopp/#1/<#2>{\ifnextchar^{\htoppp/#1/<#2>}{\htoppp/#1/<#2>^{}}}%
% \def\htoop/#1/{\ifnextchar<{\htopp/#1/}{\htopp/#1/<0>}}%
% \def\hto{\ifnextchar/{\htoop}{\htoop/>/}}%

% LABELED ISOMORPHISMS
\def\xiso#1{\mathrel{\mathrlap{\smash{\xto[\smash{\raisebox{1.3mm}{$\scriptstyle\sim$}}]{#1}}}\hphantom{\xto{#1}}}}
\def\toiso{\xto{\smash{\raisebox{-.5mm}{$\scriptstyle\sim$}}}}
\def\otiso{\xot{\smash{\raisebox{-.5mm}{$\scriptstyle\sim$}}}}

% SHADOWS
\def\shvar#1#2{{\ensuremath{%
  \hspace{1mm}\makebox[-1mm]{$#1\langle$}\makebox[0mm]{$#1\langle$}\hspace{1mm}%
  {#2}%
  \makebox[1mm]{$#1\rangle$}\makebox[0mm]{$#1\rangle$}%
}}}
\def\sh{\shvar{}}
\def\scriptsh{\shvar{\scriptstyle}}
\def\bigsh{\shvar{\big}}
\def\Bigsh{\shvar{\Big}}
\def\biggsh{\shvar{\bigg}}
\def\Biggsh{\shvar{\Bigg}}

% TYPING JUDGMENTS
% Call this macro as \jd{x:A, y:B |- c:C}.  It adds (what I think is)
% appropriate spacing, plus auto-sized parentheses around each typing judgment.
\def\jd#1{\@jd#1\ej}
\def\@jd#1|-#2\ej{\@@jd#1,,\;\vdash\;\left(#2\right)}
\def\@@jd#1,{\@ifmtarg{#1}{\let\next=\relax}{\left(#1\right)\let\next=\@@@jd}\next}
\def\@@@jd#1,{\@ifmtarg{#1}{\let\next=\relax}{,\,\left(#1\right)\let\next=\@@@jd}\next}
% Here's a version which puts a line break before the turnstyle.
\def\jdm#1{\@jdm#1\ej}
\def\@jdm#1|-#2\ej{\@@jd#1,,\\\vdash\;\left(#2\right)}
% Make an actual comma that doesn't separate typing judgments (e.g. A,B,C : Type).
\def\cm{,}

%% SKIPIT in TikZ
% See http://tex.stackexchange.com/questions/3513/draw-only-some-segments-of-a-path-in-tikz
\long\def\my@drawfill#1#2;{%
\@skipfalse
\fill[#1,draw=none] #2;
\@skiptrue
\draw[#1,fill=none] #2;
}
\newif\if@skip
\newcommand{\skipit}[1]{\if@skip\else#1\fi}
\newcommand{\drawfill}[1][]{\my@drawfill{#1}}

%% TODO: This \autoref in TAC doesn't work with figures (and anything
%% else other than theorems).

%% Also TODO: TAC's {proof} environment always inserts "Proof" at the
%% beginning even when you give it an [argument], unlike the AMS
%% {proof} environment.

%%%% THEOREM-TYPE ENVIRONMENTS, hacked to
%%% (a) number all with the same numbers, and
%%% (b) have the right names.
%% The following code should work in TAC or out of it, and with
%% hyperref or without it.  In all cases, we use \label to label
%% things and \autoref to refer to them (ordinary \ref declines to
%% include names).  The non-hyperref label and reference hack is from
%% Mike Mandell, I believe.
\newif\ifhyperref
\@ifpackageloaded{hyperref}{\hyperreftrue}{\hyperreffalse}
\iftac
  %% In the TAC style, all theorems are actually subsections.  So
  %% the hyperref \autoref doesn't work and we have to use our own code
  %% in any case.  We also have to hook into the \state macros instead
  %% of \@thm since those are what know about the current theorem type.
  \let\your@state\state
  \def\state#1{\my@state#1}
  \def\my@state#1.{\gdef\currthmtype{#1}\your@state{#1.}}
  \let\your@staterm\staterm
  \def\staterm#1{\my@staterm#1}
  \def\my@staterm#1.{\gdef\currthmtype{#1}\your@staterm{#1.}}
  \let\defthm\newtheorem
  \def\switchtotheoremrm{\let\defthm\newtheoremrm}
  % Start out \currthmtype as empty
  \def\currthmtype{}
  % In a bit, we're going to redefine \label so that \label{athm} will
  % also make a label "label@name@athm" which is the current value of
  % \currthmtype.  Now \autoref{athm} just adds a reference to this in
  % front of the reference.
  \ifhyperref
    \def\autoref#1{\ref*{label@name@#1}~\ref{#1}}
  \else
    \def\autoref#1{\ref{label@name@#1}~\ref{#1}}
  \fi
  % This has to go AFTER the \begin{document} because apparently
  % hyperref resets the definition of \label at that point.
  \AtBeginDocument{%
    % Save the old definition of \label
    \let\old@label\label%
    % Redefine \label so that \label{athm} will also make a label
    % "label@name@athm" which is the current value of \currthmtype.
    \def\label#1{%
      {\let\your@currentlabel\@currentlabel%
        \edef\@currentlabel{\currthmtype}%
        \old@label{label@name@#1}}%
      \old@label{#1}}
  }
\else
  % In non-TAC styles, theorems have their own counters and so the
  % hyperref \autoref works, if hyperref is loaded.
  \ifhyperref
    %% If we have hyperref, then we have to make sure all the theorem
    %% types appear to use different counters so that hyperref can tell
    %% them apart.  However, we want them actually to use the same
    %% counter, so we don't have both Theorem 9.1 and Definition 9.1.
    \def\defthm#1#2#3{%
      %% Ensure all theorem types are numbered with the same counter
      \newaliascnt{#1}{thm}%
      \newtheorem{#1}[#1]{#2}%
      \aliascntresetthe{#1}%
      %% This command tells cleveref's \cref what to call things
      \crefname{#1}{#2}{#3}}
  \else
    %% Without hyperref, we have to roll our own.  This code is due to
    %% Mike Mandell.  First, all theorems can now "officially" use the
    %% same counter.
    \def\defthm#1#2{\newtheorem{#1}[thm]{#2}}
    %% Save the label- and theorem-making commands
    \ifx\SK@label\undefined\let\SK@label\label\fi
    \let\old@label\label
    \let\your@thm\@thm
    %% Save the current type of theorem whenever we start one
    \def\@thm#1#2#3{\gdef\currthmtype{#3}\your@thm{#1}{#2}{#3}}
    %% Start that out as empty
    \def\currthmtype{}
    %% Redefine \label so that \label{athm} defines, in addition to a
    %% label "athm" pointing to "9.1," a label "athm@" pointing to
    %% "Theorem 9.1."
    \def\label#1{{\let\your@currentlabel\@currentlabel\def\@currentlabel%
        {\currthmtype~\your@currentlabel}%
        \SK@label{#1@}}\old@label{#1}}
    %% Now \autoref just looks at "athm@" instead of "athm."
    \def\autoref#1{\ref{#1@}}
  \fi
\fi

\ifbeamer\else

%% Now the code that works in all cases.  Note that TAC allows the
%% optional arguments, but ignores them.  It also defines environments
%% called "theorem," etc.
\newtheorem{thm}{Theorem}[section]
\newcommand{\thmautorefname}{Theorem}
\crefname{thm}{Theorem}{Theorems}
\defthm{cor}{Corollary}{Corollaries}
\defthm{prop}{Proposition}{Propositions}
\defthm{lem}{Lemma}{Lemmas}
\defthm{sch}{Scholium}{Scholia}
\defthm{assume}{Assumption}{Assumptions}
\defthm{claim}{Claim}{Claims}
\defthm{conj}{Conjecture}{Conjectures}
\defthm{hyp}{Hypothesis}{Hypotheses}
\iftac\switchtotheoremrm\else\theoremstyle{definition}\fi
\defthm{defn}{Definition}{Definitions}
\defthm{notn}{Notation}{Notations}
\iftac\switchtotheoremrm\else\theoremstyle{remark}\fi
\defthm{rmk}{Remark}{Remarks}
\defthm{eg}{Example}{Exercises}
\defthm{egs}{Examples}{Examples}
\defthm{ex}{Exercise}{Exercises}
\defthm{ceg}{Counterexample}{Counterexamples}

\let\autoref\cref

% Display format for sections
\crefformat{section}{\S#2#1#3}
\Crefformat{section}{Section~#2#1#3}
\crefrangeformat{section}{\S\S#3#1#4--#5#2#6}
\Crefrangeformat{section}{Sections~#3#1#4--#5#2#6}
\crefmultiformat{section}{\S\S#2#1#3}{ and~#2#1#3}{, #2#1#3}{ and~#2#1#3}
\Crefmultiformat{section}{Sections~#2#1#3}{ and~#2#1#3}{, #2#1#3}{ and~#2#1#3}
\crefrangemultiformat{section}{\S\S#3#1#4--#5#2#6}{ and~#3#1#4--#5#2#6}{, #3#1#4--#5#2#6}{ and~#3#1#4--#5#2#6}
\Crefrangemultiformat{section}{Sections~#3#1#4--#5#2#6}{ and~#3#1#4--#5#2#6}{, #3#1#4--#5#2#6}{ and~#3#1#4--#5#2#6}

% Display format for appendices
\crefformat{appendix}{Appendix~#2#1#3}
\Crefformat{appendix}{Appendix~#2#1#3}
\crefrangeformat{appendix}{Appendices~#3#1#4--#5#2#6}
\Crefrangeformat{appendix}{Appendices~#3#1#4--#5#2#6}
\crefmultiformat{appendix}{Appendices~#2#1#3}{ and~#2#1#3}{, #2#1#3}{ and~#2#1#3}
\Crefmultiformat{appendix}{Appendices~#2#1#3}{ and~#2#1#3}{, #2#1#3}{ and~#2#1#3}
\crefrangemultiformat{appendix}{Appendices~#3#1#4--#5#2#6}{ and~#3#1#4--#5#2#6}{, #3#1#4--#5#2#6}{ and~#3#1#4--#5#2#6}
\Crefrangemultiformat{appendix}{Appendices~#3#1#4--#5#2#6}{ and~#3#1#4--#5#2#6}{, #3#1#4--#5#2#6}{ and~#3#1#4--#5#2#6}

\fi % not beamer

% How to get QED symbols inside equations at the end of the statements
% of theorems.  AMS LaTeX knows how to do this inside equations at the
% end of *proofs* with \qedhere, and at the end of the statement of a
% theorem with \qed (meaning no proof will be given), but it can't
% seem to combine the two.  Use this instead.
\def\thmqedhere{\expandafter\csname\csname @currenvir\endcsname @qed\endcsname}

% Number numbered lists as (i), (ii), ...
\ifbeamer\else
  \renewcommand{\theenumi}{(\roman{enumi})}
  \renewcommand{\labelenumi}{\theenumi}
\fi

% Left margins for enumitem
\ifbeamer\else
  \setitemize[1]{leftmargin=2em}
  \setenumerate[1]{leftmargin=*}
\fi

% Also number formulas with the theorem counter
\iftac
  \let\c@equation\c@subsection
\else
  \let\c@equation\c@thm
\fi
\ifbeamer\else\numberwithin{equation}{section}\fi

% % Only show numbers for equations that are actually referenced (or
% % whose tags are specified manually) - uses mathtools.  All equations
% % need to be referenced with \eqref, not \ref, for this to work!
% \@ifpackageloaded{mathtools}{\mathtoolsset{showonlyrefs,showmanualtags}}{}

% GREEK LETTERS, ETC.
\alwaysmath{alpha}
\alwaysmath{beta}
\alwaysmath{gamma}
\alwaysmath{Gamma}
\alwaysmath{delta}
\alwaysmath{Delta}
\alwaysmath{epsilon}
\mdef\ep{\varepsilon}
\alwaysmath{zeta}
\alwaysmath{eta}
\alwaysmath{theta}
\alwaysmath{Theta}
\alwaysmath{iota}
\alwaysmath{kappa}
\alwaysmath{lambda}
\alwaysmath{Lambda}
\alwaysmath{mu}
\alwaysmath{nu}
\alwaysmath{xi}
\alwaysmath{pi}
\alwaysmath{rho}
\alwaysmath{sigma}
\alwaysmath{Sigma}
\alwaysmath{tau}
\alwaysmath{upsilon}
\alwaysmath{Upsilon}
\alwaysmath{phi}
\alwaysmath{Pi}
\alwaysmath{Phi}
\mdef\ph{\varphi}
\alwaysmath{chi}
\alwaysmath{psi}
\alwaysmath{Psi}
\alwaysmath{omega}
\alwaysmath{Omega}
\let\al\alpha
\let\be\beta
\let\gm\gamma
\let\Gm\Gamma
\let\de\delta
\let\De\Delta
\let\si\sigma
\let\Si\Sigma
\let\om\omega
\let\ka\kappa
\let\la\lambda
\let\La\Lambda
\let\ze\zeta
\let\th\theta
\let\Th\Theta
\let\vth\vartheta
\let\Om\Omega

%% Include or exclude solutions
% This code is basically copied from version.sty, except that when the
% solutions are included, we put them in a `proof' environment as
% well.  To include solutions, say \includesolutions; to exclude them
% say \excludesolutions.
% \begingroup
% 
% \catcode`{=12\relax\catcode`}=12\relax%
% \catcode`(=1\relax \catcode`)=2\relax%
% \gdef\includesolutions(\newenvironment(soln)(\begin(proof)[Solution])(\end(proof)))%
% \gdef\excludesolutions(%
%   \gdef\soln(\@bsphack\catcode`{=12\relax\catcode`}=12\relax\soln@NOTE)%
%   \long\gdef\soln@NOTE##1\end{soln}(\solnEND@NOTE)%
%   \gdef\solnEND@NOTE(\@esphack\end(soln))%
% )%
% \endgroup

%%% Dependent products %%%
\def\prdsym{\textstyle\prod}
%% Call the macro like \prd{x,y:A}{p:x=y} with any number of
%% arguments.  Make sure that whatever comes *after* the call doesn't
%% begin with an open-brace, or it will be parsed as another argument.
% Currently the macro is configured to produce
%     {\textstyle\prod}(x:A) \; {\textstyle\prod}(y:B),\ 
% in display-math mode, and
%     \prod_{(x:A)} \prod_{y:B}
% in text-math mode.
\def\prd#1{\@ifnextchar\bgroup{\prd@parens{#1}}{\@ifnextchar\sm{\prd@parens{#1}\@eatsm}{\prd@noparens{#1}}}}
\def\prd@parens#1{\@ifnextchar\bgroup%
  {\mathchoice{\@dprd{#1}}{\@tprd{#1}}{\@tprd{#1}}{\@tprd{#1}}\prd@parens}%
  {\mathchoice{\@dprd{#1}}{\@tprd{#1}}{\@tprd{#1}}{\@tprd{#1}}}}
\def\@eatsm\sm{\sm@parens}
\def\prd@noparens#1{\mathchoice{\@dprd@noparens{#1}}{\@tprd{#1}}{\@tprd{#1}}{\@tprd{#1}}}
% Helper macros for three styles
\def\lprd#1{\@ifnextchar\bgroup{\@lprd{#1}\lprd}{\@@lprd{#1}}}
\def\@lprd#1{\mathchoice{{\textstyle\prod}}{\prod}{\prod}{\prod}({\textstyle #1})\;}
\def\@@lprd#1{\mathchoice{{\textstyle\prod}}{\prod}{\prod}{\prod}({\textstyle #1}),\ }
\def\tprd#1{\@tprd{#1}\@ifnextchar\bgroup{\tprd}{}}
\def\@tprd#1{\mathchoice{{\textstyle\prod_{(#1)}}}{\prod_{(#1)}}{\prod_{(#1)}}{\prod_{(#1)}}}
\def\dprd#1{\@dprd{#1}\@ifnextchar\bgroup{\dprd}{}}
\def\@dprd#1{\prod_{(#1)}\,}
\def\@dprd@noparens#1{\prod_{#1}\,}

%%% Lambda abstractions.
% Each variable being abstracted over is a separate argument.  If
% there is more than one such argument, they *must* be enclosed in
% braces.  Arguments can be untyped, as in \lam{x}{y}, or typed with a
% colon, as in \lam{x:A}{y:B}. In the latter case, the colons are
% automatically noticed and (with current implementation) the space
% around the colon is reduced.  You can even give more than one variable
% the same type, as in \lam{x,y:A}.
\def\lam#1{{\lambda}\@lamarg#1:\@endlamarg\@ifnextchar\bgroup{.\,\lam}{.\,}}
\def\@lamarg#1:#2\@endlamarg{\if\relax\detokenize{#2}\relax #1\else\@lamvar{\@lameatcolon#2},#1\@endlamvar\fi}
\def\@lamvar#1,#2\@endlamvar{(#2\,{:}\,#1)}
% \def\@lamvar#1,#2{{#2}^{#1}\@ifnextchar,{.\,{\lambda}\@lamvar{#1}}{\let\@endlamvar\relax}}
\def\@lameatcolon#1:{#1}
\let\lamt\lam
% This version silently eats any typing annotation.
\def\lamu#1{{\lambda}\@lamuarg#1:\@endlamuarg\@ifnextchar\bgroup{.\,\lamu}{.\,}}
\def\@lamuarg#1:#2\@endlamuarg{#1}

%%% Dependent products written with \forall, in the same style
\def\fall#1{\forall (#1)\@ifnextchar\bgroup{.\,\fall}{.\,}}

%%% Existential quantifier %%%
\def\exis#1{\exists (#1)\@ifnextchar\bgroup{.\,\exis}{.\,}}

%%% Dependent sums %%%
\def\smsym{\textstyle\sum}
% Use in the same way as \prd
\def\sm#1{\@ifnextchar\bgroup{\sm@parens{#1}}{\@ifnextchar\prd{\sm@parens{#1}\@eatprd}{\sm@noparens{#1}}}}
\def\sm@parens#1{\@ifnextchar\bgroup%
  {\mathchoice{\@dsm{#1}}{\@tsm{#1}}{\@tsm{#1}}{\@tsm{#1}}\sm@parens}%
  {\mathchoice{\@dsm{#1}}{\@tsm{#1}}{\@tsm{#1}}{\@tsm{#1}}}}
\def\@eatprd\prd{\prd@parens}
\def\sm@noparens#1{\mathchoice{\@dsm@noparens{#1}}{\@tsm{#1}}{\@tsm{#1}}{\@tsm{#1}}}
\def\lsm#1{\@ifnextchar\bgroup{\@lsm{#1}\lsm}{\@@lsm{#1}}}
\def\@lsm#1{\mathchoice{{\textstyle\sum}}{\sum}{\sum}{\sum}({\textstyle #1})\;}
\def\@@lsm#1{\mathchoice{{\textstyle\sum}}{\sum}{\sum}{\sum}({\textstyle #1}),\ }
\def\tsm#1{\@tsm{#1}\@ifnextchar\bgroup{\tsm}{}}
\def\@tsm#1{\mathchoice{{\textstyle\sum_{(#1)}}}{\sum_{(#1)}}{\sum_{(#1)}}{\sum_{(#1)}}}
\def\dsm#1{\@dsm{#1}\@ifnextchar\bgroup{\dsm}{}}
\def\@dsm#1{\sum_{(#1)}\,}
\def\@dsm@noparens#1{\sum_{#1}\,}


%%% Definitional equality (used infix) %%%
\newcommand{\jdeq}{\equiv}      % An equality judgment
\let\judgeq\jdeq
%\newcommand{\defeq}{\coloneqq}  % An equality currently being defined
\newcommand{\defeq}{\vcentcolon\equiv}  % A judgmental equality currently being defined

%%% Bracket/squash/truncation types %%%
\newcommand{\trunc}[2]{\mathopen{}\left\Vert #2\right\Vert_{#1}\mathclose{}}
\newcommand{\ttrunc}[2]{\bigl\Vert #2\bigr\Vert_{#1}}
\newcommand{\Trunc}[2]{\Bigl\Vert #2\Bigr\Vert_{#1}}
\newcommand{\truncf}[1]{\Vert -\Vert_{#1}}
\newcommand{\tproj}[3][]{\mathopen{}\left|#3\right|_{#2}^{#1}\mathclose{}}
\newcommand{\tprojf}[1]{|-|_{#1}}
\newcommand{\brck}[1]{\trunc{}{#1}}
\newcommand{\bbrck}[1]{\ttrunc{}{#1}}
\newcommand{\Brck}[1]{\Trunc{}{#1}}
\newcommand{\bproj}[1]{\tproj{}{#1}}
\newcommand{\bprojf}{\tprojf{}}

% Transport
\newcommand{\trans}[2]{\ensuremath{{#1}_{*}\mathopen{}\left({#2}\right)\mathclose{}}\xspace}

% Big parentheses
\newcommand{\Parens}[1]{\Bigl(#1\Bigr)}
\let\UU\cU

\newcommand{\ty}{\;\textsf{type}}
\newcommand{\prp}{\;\textsf{prop}}

\newcommand{\fc}[1]{FOLDS-pre$^{#1}$category}
\newcommand{\fcs}[1]{FOLDS-pre$^{#1}$categories}

\makeatother

% Local Variables:
% mode: latex
% TeX-master: "ufolds"
% End:

\UseAllTwocells
\newcommand{\blank}{\mathord{\hspace{1pt}\text{--}\hspace{1pt}}}

\usetheme[height=.7cm]{Rochester}
\useinnertheme{circles}
%\useoutertheme[compress,subsection=false]{miniframes}
\usecolortheme{rose}
\setbeamertemplate{navigation symbols}{}

\title{Univalent FOLDS}
\author[author]{Michael Shulman\inst{1} \and Benedikt Ahrens\inst{2} \and Paige North\inst{3}}
\institute{\inst{1} (University of San Diego)
  \and \inst{2} (Centre International de Mathématiques et Informatique)
  \and \inst{3} (University of Cambridge)}
\date{}

\AtBeginSection[]
{
  \begin{frame}<beamer>{Outline}
    \tableofcontents[currentsection]
  \end{frame}
}

\begin{document}

\begin{frame}
  \maketitle
\end{frame}

\section{The Problem: Equivalence invariance}
\label{sec:invariance}

\begin{frame}
  \frametitle{Equivalence Invariance}
  \begin{block}{Question}
    We generally work with categories ``up to equivalence''.  How do we know that a true statement about one category is also true about any \alert<1>{equivalent} category?
  \end{block}
  \pause
  Well\dots it's not.
  \begin{itemize}
  \item<+-> The category of sets has exactly one initial object, but plenty of \alert<2>{equivalent} categories (e.g.\ $\mathbf{CABA}^{\mathrm{op}}$) have more than one.
  \item<+-> \structure{A:} That's stupid.  Every \alert<3>{sensible} statement is invariant under equivalence.
  \item<+-> \structure{Q:} What does ``sensible'' mean?
  \item<+-> \structure{A:} \dots I know it when I see it.
  \end{itemize}
\end{frame}

\begin{frame}
  \frametitle{Why it matters}
  When is ``I know it when I see it'' not good enough?
  \begin{enumerate}
  \item When the notion of ``equivalence'' is less obvious (e.g.\ no AC)
  \item When syntax matters, e.g.\ internalizing in a topos
  \item Generalizing to higher categorical structures
  \item Formalizing category theory in a computer
  \end{enumerate}
  \pause
  \begin{block}{A possible solution (Freyd, Blanc, Makkai)}
    \begin{enumerate}
    \item Prove that any statement \alert{with a certain syntactic form} is invariant under equivalence.
    \item Non-invariance comes from using \alert{equality of objects} (rather than isomorphism).
  \end{enumerate}
  \end{block}
\end{frame}

\begin{frame}
  \frametitle{Dependent types}
  \begin{block}{Problem}
    The \alert<1>{definition} of category uses equality of objects:
    \begin{center}
      ``If $s(g)$ is \alert<1>{equal to} $t(f)$, then $g\circ f$ exists.''
    \end{center}
  \end{block}
  \begin{block}{Solution}
    Use a logic of {dependent types}, in which $s(g)=t(f)$ is encoded by what {type of thing} $f$ and $g$ are.
  \end{block}
  A category consists of
  \begin{itemize}
  \item A collection $O$ of objects
  \item For each $x,y:O$, a collection $A(x,y)$ of arrows
  \item For each $x,y,z:O$ and each $f:A(x,y)$ and $g:A(y,z)$, a composite $g\circ f: A(x,z)$
  \item For each $x:O$, an identity $1_x:A(x,x)$
  \item \dots
  \end{itemize}
\end{frame}

\begin{frame}
  \frametitle{Syntactic invariance}
  \begin{theorem}[Freyd 76, Blanc 78, Makkai 95]
    A {property} of categories is invariant under equivalence iff it can be expressed in this dependently typed language, using equality for arrows but not for objects.
  \end{theorem}
  This is great:
  \begin{enumerate}
  \item It's an easily checkable syntactic criterion.
  \item It's really just the language we usually use anyway!
  \end{enumerate}
  \pause
  \begin{block}{Goal}
    \begin{enumerate}
    \item Generalize this to higher categorical structures\dots
    \item \dots in a way that a computer can understand.
    \end{enumerate}
  \end{block}
\end{frame}

\section{Higher categories: FOLDS}
\label{sec:folds}

\begin{frame}[t]
  \frametitle{Dependent types for higher categories}
  \begin{block}{Observation 1 (Makkai 95)}
    Every (higher) categorical structure has a natural dependently typed language.
  \end{block}
  \begin{itemize}
  \item \structure{2-categories:} objects $O$, arrows $A(x,y)$, 2-arrows $W_{x,y}(f,g)$
    \[\xymatrix{x \rtwocell^f_g & y}\]
  \item \structure{3-categories:} objects $O$, arrows $A(x,y)$, 2-arrows $W_{x,y}(f,g)$, 3-arrows $H_{x,y,f,g}(\alpha,\beta)$
  \item \structure{double categories:} objects $O$, horizontal arrows $H(x,y)$, vertical arrows $V(x,y)$, squares $S_{x,y,z,w}(f,g,h,k)$
    \[\xymatrix@-1pc{x \ar[r]^f \ar[d]_h \drtwocell\omit & y \ar[d]^k \\ z \ar[r]_{g} & w}\]
  \end{itemize}
\end{frame}

\begin{frame}[t]
  \frametitle{Operations as types}
  \begin{block}{Observation 2}
    We can also express \alert<1>{operations} as dependent types.
  \end{block}
  A category consists of
  \begin{itemize}
    \only<1>{
  \item A collection $O$ of objects
  \item For each $x,y:O$, a collection $A(x,y)$ of arrows
  \item For each $x,y,z:O$, $f:A(x,y)$, $g:A(y,z)$, and $h:A(x,z)$, a collection $T_{x,y,z}(f,g,h)$ of \alert<1>{assertions that $h$ is $g\circ f$}.
  \item For each $x:O$ and $f:A(x,y)$, a collection $I_x(f)$ of \\\alert<1>{assertions that $f$ is $1_x$}.
  \item \dots
  }
  \only<2-3>{
  \item \dots $O$, $A$, $T$, $I$, \dots
  \item For any $f:A(x,y)$, and $g:A(y,z)$, there exist $h:A(x,z)$ and $t:T_{x,y,z}(f,g,h)$.
  \item For any $t_1:T_{x,y,z}(f,g,h_1)$ and $t_2:T_{x,y,z}(f,g,h_2)$, $h_1 = h_2$.
  \item \dots
    \pause\pause
  \item For any $t_1,t_2:T_{x,y,z}(f,g,h)$, we have $t_1 = t_2$\\ ($T$ is a mere relation).
  \item For any $x,f$ with $i_1,i_2 : I_x(f)$ we have $i_1=i_2$\\ ($I$ is a mere relation).
  }
  \end{itemize}
\end{frame}

\begin{frame}[t]
  \frametitle{Equality as a type}
  \begin{block}{Representing equality}
    We include \alert{equality of morphisms} as an explicit relation.
  \end{block}
  A category consists of
  \begin{itemize}
  \item \dots $O$, $A$, $T$, $I$, \dots
  \item For each $x,y:O$ and $f,g:A(x,y)$, a collection $E_{x,y}(f,g)$ of \alert{assertions that $f=g$}.
  \item \dots
    \pause
  \item For any $t_1:T_{x,y,z}(f,g,h_1)$ and $t_2:T_{x,y,z}(f,g,h_2)$, we have $e : E_{x,z}(h_1,h_2)$.
    % \pause
  \item $T$, $I$, and now also $E$ are mere relations.
  \item $E$ is an equivalence relation, and a congruence for $T$ and $I$.
  \end{itemize}
\end{frame}

\begin{frame}
  \frametitle{FOLDS-signatures}
  We describe a dependent type structure with a \alert<1>{simple category} $\Sigma$.
  \[
  \vcenter{\xymatrix{
      T \ar[dr] \ar@<1mm>[dr] \ar@<-1mm>[dr] & I \ar[d] & E \ar@<1mm>[dl] \ar@<-1mm>[dl] \\
      & A \ar@<1mm>[d] \ar@<-1mm>[d] \\
      & O}}
  \]
  \pause
  \begin{itemize}
  \item A \alert<2>{structure} is a {diagram} $X$ on $\Sigma$.
  \item The objects $K$ of $\Sigma$ are called \alert<2>{kinds}.
  \item The \alert<2>{matching object} $M_K X = \lim_{(K\to K') \neq \mathsf{id}} X_{K'}$ of $X$ and $K$ encodes the possible dependencies of $x\in X_K$.
  \item The fibers of $X_K\to M_K X$ are \alert<2>{types}, e.g.\ $A(a,b)$, $T_{x,y,z}(f,g,h)$.
  \end{itemize}
  \pause
  \alert{FOLDS} = \alert{F}irst-\alert{O}rder \alert{L}ogic with \alert{D}ependent \alert{S}orts
\end{frame}

\begin{frame}[t]
  \frametitle{Equivalences for higher categories}
  \begin{block}{Observation 4 (Makkai 95)}
    Every (higher) categorical structure has an intrinsic notion of \alert<1>{equivalence} arising from its dependently typed language.
  \end{block}
  \begin{enumerate}
\only<1>{\item Categories $C$ and $D$ are equivalent iff there is a span
    \[ C \ot X \to D\]
    of surjective equivalences.
  }\only<2->{\setcounter{enumi}{1}
  \item A \only<2>{functor}\only<3>{\alert{map of structures}} is $F:C\to D$ is a surjective equivalence iff
    \begin{itemize}
    \item For each $y:O_D$, there exists $x:O_C$ such that $F(x)=y$ (\alert<2>{surjective on objects}).
    \item For each $x,y:O_C$ and $g:A_D(Fx,Fy)$, there exists $f:A_C(x,y)$ such that $F(f)=g$ (\alert<2>{full}).
    \item For $x,y:O_C$ and $f,g:A_C(x,y)$ and $e_D:E_{D,(Fx,Fy)}(Ff,Fg)$, there exists $e_C:E_{C,(x,y)}(f,g)$ such that $Fe_C=e_D$ (\alert<2>{faithful}).
      \only<3->{
    \item For $x,y,z,f,g,h$ and $t_D:T_{D,(Fx,Fy,Fz)}(Ff,Fg,Fh)$, there exists $t_C:T_{C,(x,y,z)}(f,g,h)$ with $Ft_C = t_D$ (\alert{preserves composition}).
    \item For $x,f$ and $i_D:I_{D,Fx}(Ff)$, there exists $i_C:I_{C,x}(f)$ such that $Fi_C = i_D$ (\alert{preserves identities}).
      }
    \end{itemize}}
  \end{enumerate}
\end{frame}

\begin{frame}[t]
  \frametitle{Equivalences for higher categories}
  \begin{block}{Observation 4 (Makkai 95)}
    Every (higher) categorical structure has an intrinsic notion of {equivalence} arising from its dependently typed language.
  \end{block}
  \begin{enumerate}
  \item A morphism $X\to Y$ of FOLDS-structures is \alert<1>{very surjective} (a.k.a.\ \structure<1>{Reedy surjective} or a \structure<1>{trivial fibration}) if each ``matching map'' $X_K \to Y_K \times_{M_K Y} M_K X$ is surjective.
  \item Structures $C$ and $D$ are \alert<1>{FOLDS-equivalent} if there is a span \( C \ot X \to D\) of very surjective maps.
  \end{enumerate}
  \pause
  \begin{theorem}[Makkai 95]
    A {statement} about FOLDS-structures is invariant under FOLDS-equivalence \alert<2>{if and only if} it can be expressed in their dependently typed language, using equality only at ``top level''.
  \end{theorem}
\end{frame}

\begin{frame}
  \frametitle{FOLDS redux}
  \begin{block}{Pros}
    \begin{itemize}
    \item Applies generally to any categorical structure.
    \item An easily checkable syntactic criterion.
    \item Really just the language we usually use anyway.
    \end{itemize}
  \end{block}
  % \pause
  \begin{block}{Cons}
    \begin{itemize}
    \item The ``mere relations'' ($T$, $I$, $E$) play a curiously special role.
    \item Need to include equality $E$ as a relation, but then not all structures are ``standard'' (interpret $E$ by ``actual'' equality).
    \item FOLDS-equivalence is complicated and ``large''.
    \end{itemize}
  \end{block}
\end{frame}

\section{Reducing to groupoids: Univalent categories}
\label{sec:univalence}

\begin{frame}
  \frametitle{Reducing to groupoids}
  \begin{block}{Idea}
    \begin{enumerate}
    \item Define categories as \alert<1>{structured groupoids} rather than structured sets.
    \item An equivalence of categories is a levelwise equivalence of groupoids.
    \end{enumerate}
  \end{block}
  Why?
  \begin{itemize}
  \item Tools from homotopy theory make (higher) groupoids easier to work with.
  \item (Higher) groupoids are a common reduct for all higher categorical structures.
  \item Levelwise equivalences are easier to work with.
  \item Homotopy type theory: computers understand higher groupoids!
  \end{itemize}
\end{frame}

\begin{frame}
  \frametitle{Saturating categories}
  For any category $C$, its \alert<1>{saturation} is an internal category $\Chat = (A\toto O)$ in groupoids:
  \begin{itemize}
  \item $O$ is the groupoid of objects of $C$ and isomorphisms in $C$.
  \item $A$ is the groupoid of morphisms of $C$ and their isomorphisms (commutative squares).
  \end{itemize}
  Equivalently, $\Chat$ is the double category of objects, morphisms, isomorphisms, and commutative squares in $C$.
  \pause
  \begin{block}{Fact}
    $C\to D$ is an equivalence iff $\Chat \to \Dhat$ is an \alert{equivalence on underlying groupoids} (i.e.\ $O_{\chat} \to O_{\Dhat}$ and $A_{\Chat} \to A_{\Dhat}$ are equivalences).
  \end{block}
\end{frame}

\begin{frame}
  \frametitle{Categories in groupoids}
  \begin{theorem}
    An internal category $A\toto O$ in groupoids is of the form $\Chat$ iff
    \begin{enumerate}
    \item The map $A\to O\times O$ is a discrete fibration, and
    \item The canonical map
    \[ \hom_O(x,y) \to \mathsf{iso}(x,y) \]
    is an isomorphism for all $x,y\in \mathsf{ob}(O)$.
    \end{enumerate}
  \end{theorem}
  \begin{itemize}
  \item $\hom_O(x,y)=$ morphisms in the groupoid $O$
  \item $\mathsf{iso}(x,y)=$ objects of the groupoid $A$ that are isomorphisms from $x$ to $y$ for the category structure on $A\toto O$.
  \end{itemize}
  Such a category in groupoids is called \alert{univalent} (or ``saturated'' or ``Rezk-complete'').
\end{frame}

% \begin{frame}
%   \frametitle{Univalent foundations}
%   \begin{block}{Univalent foundations}
%     An alternative to set-theoretic foundations for mathematics in which
%     \begin{itemize}
%     \item The basic objects are not sets but $\infty$-groupoids (called \alert<1>{(homotopy) types})
%       \pause
%     \item Instead of the \alert<2>{proposition} $x=y$ of set theory, we have for any type $X$ and elements $x,y:X$, \alert<2>{another type} ``$x=y$''.
%       \pause
%     \item The elements of $x=y$ are called \alert<3>{morphisms}, \alert<3>{paths}, or \alert<3>{identifications} from $x$ to $y$.
%       \pause
%     \item There is a type $\mathcal{U}$ whose elements are types, such that for $A,B:\mathcal{U}$ the type $A=B$ consists of \alert<4>{equivalences} from $A$ to $B$ (the \alert<4>{univalence axiom})
%       \pause
%     \item There is an operation that automatically transports all statements \alert<5>{or constructions} along an identification $p:x=y$.
%     \end{itemize}
%   \end{block}
% \end{frame}

\begin{frame}
  \frametitle{Univalent FOLDS}
  Consider ``Reedy fibrant'' FOLDS-structures in $\infty$-groupoids.
  \begin{itemize}
  \item Each $O$, $A$, $T$, etc.\ is an $\infty$-groupoid.
  \item Each ``dependency-assigning map'' $K\to M_K$ is a fibration.\\
    (E.g.\ $A \to O\times O$ or $E\to A\times_{O\times O} A$)
  \end{itemize}
  \begin{theorem}[in progress]
    There is a notion of \alert{univalence} for FOLDS-structures in $\infty$-groupoids, such that
    \begin{enumerate}
    \item Two univalent structures are FOLDS-equivalent iff they are levelwise equivalent on underlying $\infty$-groupoids.
    \item In a univalent structure, the top sorts (e.g.\ $T$, $I$, $E$) are {automatically} mere relations.
    \item A univalent structure is automatically standard ($E$ is actual equality).
    \end{enumerate}
  \end{theorem}
  % \structure{NB:} The hard part is defining ``univalent''.
  % \pause\medskip
  % \begin{block}{Higher category theory in UF}
  %   \begin{itemize}
  %   \item We can define \alert{any higher-categorical structure} in univalent foundations, such that $C=D$ consists of the appropriate sort of  equivalences.
  %   \item In particular, any statement \alert{or construction} can automatically be transported along such equivalences.
  % \end{itemize}
  % \end{block}
\end{frame}

\section{Univalent FOLDS by example}
\label{sec:univalent-folds-eg}

\begin{frame}
  \frametitle{Univalent FOLDS-structures}
  \begin{definition}
    A FOLDS-structure in $\infty$-groupoids is \alert<1>{univalent} if for every kind $K$, $\vec{k}\in M_K$, and $a,b:K(\vec{k})$, the canonical map
    \[ \hom_{K(\vec{k})}(a,b) \to \mathsf{iso}(a,b) \]
    is an equivalence.
  \end{definition}
  \pause
  \begin{itemize}
  \item $\hom_{K(\vec{k})}(a,b)=$ the $\infty$-groupoid of maps in the fiber of the $\infty$-groupoid $K$ over $\vec{k} \in M_K$.
  \item $\mathsf{iso}(a,b)=$ the $\infty$-groupoid of \alert{FOLDS-isomorphisms} $a$ to $b$.
  \end{itemize}
  What is a FOLDS-isomorphism?
\end{frame}

\begin{frame}[t]
  \frametitle{FOLDS-isomorphisms for categories}
  \begin{block}{Idea}
    A FOLDS-isomorphism between $a,b:K(\vec{k})$ is\\\alert<1>{a coherent set of equivalences replacing $a$ by $b$ in the types of anything in $C$ that depends on one or more elements of $K$}.
  \end{block}
  \pause
  \begin{example}
  For objects $a,b:O$ of a FOLDS-category $(O,A,T,I,E,\dots)$, a FOLDS-isomorphism is
  \begin{enumerate}
  \item For each $x:O$, an isomorphism $\phi_{x\bullet} : A(x,a) \cong A(x,b)$,
  \item For each $z:O$, an isomorphism $\phi_{\bullet z} : A(a,z) \cong A(b,z)$, and
  \item An isomorphism $\phi_{\bullet\bullet} : A(a,a) \cong A(b,b)$,
  \item Respecting composites, identities, and equality ($T$, $I$, and $E$)\\
    \textit{(to be continued \dots)}
  \end{enumerate}
\end{example}
  % \pause
  % This is equivalent to an ordinary isomorphism $x\cong y$, by a ``two-sided'' form of the Yoneda lemma.
\end{frame}

\begin{frame}[t]
  \frametitle{FOLDS-isomorphisms for categories}
  \[
  \begin{array}{rcl}
    \alert<2>{T_{x,y,a}(f,g,h)} & \alert<2>{\Leftrightarrow}& \alert<2>{T_{x,y,b}(f,\phi_{y\bullet}(g),\phi_{x\bullet}(h))}\\
    \alert<4>{T_{x,a,z}(f,g,h)} &\alert<4>{\Leftrightarrow}& \alert<4>{T_{x,b,z}(\phi_{x\bullet}(f),\phi_{\bullet z}(g),h)}\\
    \alert<3>{T_{a,z,w}(f,g,h)} &\alert<3>{\Leftrightarrow}& \alert<3>{T_{b,z,w}(\phi_{\bullet z}(f),g,\phi_{\bullet w}(h))}\\
    \alert<5>{T_{x,a,a}(f,g,h)} &\alert<5>{\Leftrightarrow}& \alert<5>{T_{x,b,b}(\phi_{x\bullet}(f),\phi_{\bullet\bullet}(g),\phi_{x\bullet}(h))}\\
    \alert<5>{T_{a,x,a}(f,g,h)} &\alert<5>{\Leftrightarrow}& \alert<5>{T_{b,x,b}(\phi_{\bullet x}(f),\phi_{x\bullet}(g),\phi_{\bullet\bullet}(h))}\\
    \alert<5>{T_{a,a,x}(f,g,h)} &\alert<5>{\Leftrightarrow}& \alert<5>{T_{b,b,x}(\phi_{\bullet\bullet}(f),\phi_{\bullet x}(g),\phi_{\bullet x}(h))}\\
    \alert<6>{T_{a,a,a}(f,g,h)} &\alert<6>{\Leftrightarrow}& \alert<6>{T_{b,b,b}(\phi_{\bullet\bullet}(f),\phi_{\bullet\bullet}(g),\phi_{\bullet\bullet}(h))}\\
    \alert<6>{I_{a,a}(f) }& \alert<6>{\Leftrightarrow}& \alert<6>{I_{b,b}(\phi_{\bullet\bullet}(f))}\\
    \alert<6>{E_{x,a}(f,g) }&\alert<6>{\Leftrightarrow}& \alert<6>{E_{x,b}(\phi_{x\bullet }(f),\phi_{x\bullet}(g))}\\
    \alert<6>{E_{a,x}(f,g) }&\alert<6>{\Leftrightarrow}& \alert<6>{E_{b,x}(\phi_{\bullet x}(f),\phi_{\bullet x}(g))}\\
    \alert<6>{E_{a,a}(f,g) }&\alert<6>{\Leftrightarrow}& \alert<6>{E_{b,b}(\phi_{\bullet \bullet}(f),\phi_{\bullet \bullet}(g))}
  \end{array}
  \]
  \vspace{1mm}
  \begin{center}
    \only<2>{The $\phi_{x\bullet}$ are a natural isomorphism.}
    \only<3>{The $\phi_{\bullet x}$ are a natural isomorphism.}
    \only<4>{The $\phi_{x\bullet}$ and $\phi_{\bullet x}$ arise from the same iso $a\cong b$.}
    \only<5>{$\phi_{\bullet\bullet}$ is conjugation by this iso.\\
      (Any one of these ensures this, and the others follow.)}
    \only<6>{The rest are then automatic.}
  \end{center}
\end{frame}

\begin{frame}[t]
  \frametitle{FOLDS-isomorphisms for categories}
  \begin{example}
    For objects $a,b:O$ of a FOLDS-category $(O,A,T,I,E,\dots)$, a FOLDS-isomorphism is
    \begin{enumerate}
    \item For each $x:O$, an isomorphism $\phi_{x\bullet} : A(x,a) \cong A(x,b)$,
    \item For each $z:O$, an isomorphism $\phi_{\bullet z} : A(a,z) \cong A(b,z)$, and
    \item An isomorphism $\phi_{\bullet\bullet} : A(a,a) \cong A(b,b)$,
    \item Respecting composites, identities, and equality ($T$, $I$, and $E$).
    \end{enumerate}
  \end{example}
  \begin{theorem}[Two-sided Yoneda lemma]
    FOLDS-isomorphisms are equivalent to ordinary isos $a\cong b$.
  \end{theorem}
\end{frame}

\begin{frame}
  \frametitle{FOLDS-univalence for objects in categories}
  \begin{block}{That means}
    Our proposed general ``univalence'' condition for FOLDS-structures, applied at $O$ in the signature for categories, gives the second condition $\hom_O(x,y) \toiso \mathsf{iso}(x,y)$ for univalent categories.
  \end{block}
  \begin{block}{But}
    \begin{enumerate}
    \item To be consistent, we must apply that condition at the other kinds too: $A$, $T$, $I$, $E$.
    \item Also need the first condition: $A\to O\times O$ is a discrete fibration.
    \end{enumerate}
  \end{block}
  Fortunately, these two difficulties resolve each other.
\end{frame}

\begin{frame}
  \frametitle{FOLDS-isomorphisms for operations}
  \[
  \vcenter{\xymatrix{
      % &&& \text{FOLDS} \cong\\
      \alert<1-2>{T} \ar[dr] \ar@<1mm>[dr] \ar@<-1mm>[dr] & \alert<1-2>{I} \ar[d] & \alert<1-2>{E} \ar@<1mm>[dl] \ar@<-1mm>[dl] &
      \only<1>{?}\only<2->{\text{vacuous}}\\
      & \alert<3>{A} \ar@<1mm>[d] \ar@<-1mm>[d] && \only<3>{?} \\
      & {O}&& \text{ordinary} \cong}}
  \]
  \pause
  \begin{itemize}
  \item The types $T$, $I$, and $E$ have \alert<2>{nothing} that depend on them.
  \item Thus, $\mathsf{iso}(x,y)=1$ vacuously for any $x,y$ in these types.
  \end{itemize}
  \begin{block}{Hence}
    Univalence at $T$ means any two $t_1,t_2 : T_{x,y,z}(f,g,h)$\\ (for instance) are equal, i.e.\ $T$ is a mere relation.
  \end{block}
\end{frame}

\begin{frame}
  \frametitle{FOLDS-isomorphisms for arrows}
  A FOLDS-iso between $f,g:A(x,y)$ includes, among other things,
  \begin{itemize}
  \item For any $h:A(x,y)$, we have $E_{x,y}(f,h) \iff E_{x,y}(g,h)$.
  \end{itemize}
  and therefore, since $E$ is reflexive,
  \begin{itemize}
  \item $E_{x,y}(f,g)$.
  \end{itemize}
  % \pause
  Since $E$ is a congruence for $T$ and $I$, this implies everything else.
  So \alert{a FOLDS-iso between $f,g:A(x,y)$ is the assertion $E_{x,y}(f,g)$.}
  \begin{block}{Thus}
    Univalence at $A$ means that the fibers of $A\to O\times O$ are discrete sets and their equality is $E$.
  \end{block}
  % \pause
  \structure{Note:}
    Similarly, univalence at $O$ also ensures that $O$ is a 1-groupoid, even if \textit{a priori} it was an $\infty$-groupoid.
\end{frame}

\begin{frame}
  \frametitle{FOLDS-univalence for categories}
  \begin{block}{In conclusion}
    A FOLDS-category in $\infty$-groupoids satisfies the univalence condition at all types if and only if it is a univalent 1-category in groupoids, hence of the form $\Chat$ for some ordinary category $C$.
  \end{block}
  \vspace{1cm}
  % \pause
  Is it sensible in other examples?
\end{frame}

\begin{frame}[t]
  \frametitle{$\dagger$-categories}
  \begin{definition}
    A \alert<1>{$\dagger$-category} is a category with coherent isomorphisms $(\blank)^\dagger : \hom(x,y) \to \hom(y,x)$.
  \end{definition}
  \only<1>{\[
  \vcenter{\xymatrix@R=1.5pc{
      D \ar@<0mm>[dr] \ar@<-1mm>[dr] & T \ar[d] \ar@<1mm>[d] \ar@<-1mm>[d] & I \ar[dl] & E \ar@<1mm>[dll] \ar@<0mm>[dll] \\
      & A \ar@<1mm>[d] \ar@<-1mm>[d] \\
      & O}}
  \]}
  \pause
  \begin{itemize}
  \item Now a FOLDS-isomorphism between $x,y:O$ is a \alert{unitary isomorphism} $f:x\toiso y$, i.e.\ one satisfying $f^{-1} = f^\dagger$.
    % \pause
  \item In a univalent $\dagger$-category, $O$ is the groupoid of objects and unitary isomorphisms.
  \item FOLDS-equivalences (which are underlying equivalences of univalent $\dagger$-categories) are \alert{$\dagger$-equivalences}, involving unitary natural isomorphisms.
  \end{itemize}
\end{frame}

\begin{frame}
  \frametitle{Classical first-order logic}
  Consider a many-sorted first-order theory with relations and equality:
  \[ \xymatrix{
    E_1 \ar@<1mm>[d] \ar@<-1mm>[d] & R_1 \ar[dl] \ar[dr] & E_2 \ar@<1mm>[d] \ar@<-1mm>[d] & R_2 \ar[dlll] \ar@<1mm>[dr] \ar@<-1mm>[dr] & E_3 \ar@<1mm>[d] \ar@<-1mm>[d] & R_3 \ar[dl] & \dots \\
    A_1 && A_2 && A_3 && \dots
  }\]
  \begin{itemize}
  \item Univalence at $E_i$ and $R_i$ forces them to have at most one element each, i.e.\ to be relations.
  \item Univalence at $A_i$ makes it a discrete set whose equality is $E_i$.
  \end{itemize}
  We recover classical first-order logic with equality.
\end{frame}

\begin{frame}[t]
  \frametitle{2-categories}
  What about 2-categories?
  \[
  \vcenter{\xymatrix{
      I_2 \ar[drr] & T_2 \ar[dr] \ar@<1mm>[dr] \ar@<-1mm>[dr] & E_2 \ar@<1mm>[d] \ar@<-1mm>[d] \\
      && A_2 \ar@<1mm>[d] \ar@<-1mm>[d]\\
      && A_1 \ar@<1mm>[d] \ar@<-1mm>[d] \\
      && O}}
      % T_2 \ar[dr] \ar@<1mm>[dr] \ar@<-1mm>[dr] & I_2 \ar[d] & E_2 \ar@<1mm>[dl] \ar@<-1mm>[dl] \\
      % & A_2 \ar@<1mm>[d] \ar@<-1mm>[d] \\
      % & A_1 \ar@<1mm>[d] \ar@<-1mm>[d] \\
      % & O}}
  \]
\end{frame}

\begin{frame}[t]
  \frametitle{2-categories}
  \[
  \vcenter{\xymatrix{
      I_2 \ar[drr] & T_2 \ar[dr] \ar@<1mm>[dr] \ar@<-1mm>[dr] & E_2 \ar@<1mm>[d] \ar@<-1mm>[d] \\
      && A_2 \ar@<1mm>[d] \ar@<-1mm>[d] & T_1 \; ? \ar[dl] \ar@<1mm>[dl] \ar@<-1mm>[dl]  \\
      && A_1 \ar@<1mm>[d] \ar@<-1mm>[d] \\
      && O}}
  \]
  \begin{itemize}
  \item Since $T_1 \to (A_1)^3$ is a fibration, any $t:T_1(f,g,h)$ can be transported along an iso $h \cong h'$ to yield $T_1(f,g,h')$.
  \item $t:T_1(f,g,h)$ must record an iso $h \cong g \circ f$, so $T_1$ can't be a mere relation.
  \end{itemize}
\end{frame}

\begin{frame}[t]
  \frametitle{2-categories}
  \[
  \vcenter{\xymatrix{
      I_2 \ar[drr] & T_2 \ar[dr] \ar@<1mm>[dr] \ar@<-1mm>[dr] & E_2 \ar@<1mm>[d] \ar@<-1mm>[d]
      & E_{T_1} \ar@<1mm>[d] \ar@<-1mm>[d]\\
      && A_2 \ar@<1mm>[d] \ar@<-1mm>[d] & T_1  \ar[dl] \ar@<1mm>[dl] \ar@<-1mm>[dl]  \\
      && A_1 \ar@<1mm>[d] \ar@<-1mm>[d] \\
      && O}}
  \]
  \begin{itemize}
  \item Need an equality on $T_1$ to allow it to be a set.
  \end{itemize}
\end{frame}

\begin{frame}[t]
  \frametitle{2-categories}
  \[
  \vcenter{\xymatrix{
      I_2 \ar[drr] & T_2 \ar[dr] \ar@<1mm>[dr] \ar@<-1mm>[dr] & E_2 \ar@<1mm>[d] \ar@<-1mm>[d]
      & ? \ar[dl] \ar[d]
      & E_{T_1} \ar@<1mm>[dl] \ar@<-1mm>[dl]\\
      && A_2 \ar@<1mm>[d] \ar@<-1mm>[d] & T_1  \ar[dl] \ar@<1mm>[dl] \ar@<-1mm>[dl]  \\
      && A_1 \ar@<1mm>[d] \ar@<-1mm>[d] \\
      && O}}
  \]
  \begin{itemize}
  \item Need an equality on $T_1$ to allow it to be a set.
  \item Need to relate $T_1$ to the isos in $A_2$\dots
  \end{itemize}
\end{frame}

\begin{frame}[t]
  \frametitle{2-categories}
  \begin{block}{Options}
    \begin{enumerate}
    \item Use a ``fat truncated $\Theta$''.  Need an extra \alert<1>{saturation} condition.
      \pause
    \item Use \alert{opetopes} / \alert{multitopes} \\(Baez--Dolan, Hermida--Makkai--Power).
    \end{enumerate}
  \end{block}
  \[
  \vcenter{\xymatrix{
      A_{2,0} \ar[dr]
      & A_{2,1} \ar@<1mm>[d] \ar@<-1mm>[d]
      & A_{2,2} \ar@<1mm>[dl] \ar@<-1mm>[dl] \ar[dl]
      & \cdots \ar@{}[dll]|{\cdots} \\
      & A_1 \ar@<1mm>[d] \ar@<-1mm>[d] \\
      & O}}
  \]
  \begin{itemize}
  \item Need equalities $E_{2,n}$, many compositions of different arities.
  \end{itemize}
\end{frame}

\begin{frame}
  \frametitle{$\omega$-categories}
  What about $\omega$-categories?
  \begin{columns}
    \begin{column}{2cm}
      \[
      \vcenter{\xymatrix{
          & \vdots \ar@<1mm>[d] \ar@<-1mm>[d] \\
          & A_3 \ar@<1mm>[d] \ar@<-1mm>[d] \\
          & A_2 \ar@<1mm>[d] \ar@<-1mm>[d] \\
          & A_1 \ar@<1mm>[d] \ar@<-1mm>[d] \\
          & O}}
      \]
    \end{column}
    \begin{column}{8cm}
      \begin{block}{Expectations}
        \begin{itemize}
        \item No top-level sorts $\Rightarrow$ no mere relations
        \item All types are general $\infty$-groupoids
        \item Need infinite limits to define FOLDS-isos
          \pause
        \item FOLDS-isos are \alert{not} coinductive, but downward inductive.
        \end{itemize}
      \end{block}
      What is an equivalence in an $\omega$-category?
    \end{column}
  \end{columns}
\end{frame}

\begin{frame}
  \begin{center}
    Thanks!
  \end{center}
\end{frame}

\appendix

\section{Appendix: FOLDS-isomorphisms by definition}
\label{sec:univalent-folds}

\begin{frame}
  \frametitle{Versions of the identity}
  \begin{block}{\only<1-2>{Wrong }\only<3->{Right }Idea}
    $a,b:K(\vec{k})$ in $C$ are \alert<1>{FOLDS-isomorphic} if
    \begin{itemize}
    \item \only<1>{there is a FOLDS-equivalence $C\simeq C$ taking $a$ to $b$.}\only<2>{the \alert<2>{identity} FOLDS-equivalence $C \simeq C$ takes $a$ to $b$.}\only<3->{the identity FOLDS-equivalence $C \simeq C$ \alert<3>{could} take $a$ to $b$.}
    \end{itemize}
  \end{block}
  \pause\pause\pause
  \begin{definition}
    A FOLDS-equivalence $C\simeq C$, i.e.\ a span of very-surjective maps
    \[ C \ot F \to C, \]
    is \alert{a version of the identity} if there is a common section:
    \[ \xymatrix{ &C \ar[d] \ar@{=}[dl] \ar@{=}[dr] \\ C & F \ar[l] \ar[r] & C } \]
  \end{definition}
\end{frame}

\begin{frame}
  \frametitle{Ana-equivalences of categories}
  \begin{example}
    For 1-categories, a FOLDS-equivalence $C \ot F \to D$ is an \alert<1>{ana-equivalence}:
    \begin{itemize}
    \item The objects of $F$ are ``reasons why $F(c) = d$''
    \item For every $c\in C$, there is a $d\in D$ and a reason why $F(c)=d$.
    \item For every $d\in D$, there is a $c\in C$ and a reason why $F(c)=d$.
    \item If $F(c)=d$ and $F(c')=d'$ (with given reasons), then we have an isomorphism $\hom_C(c,c') \toiso \hom_D(d,d')$.
    \item These isomorphisms are functorial.
    \end{itemize}
  \end{example}
  An endo-ana-equivalence $C \ot F \to C$ is a \structure{version of the identity} if for all $c$, there is some reason why $F(c)=c$.
\end{frame}

\begin{frame}
  \frametitle{FOLDS-isomorphic objects}
  \begin{definition}[Makkai]
    $a,b:K(\vec{k})$ in $C$ are \alert<1>{FOLDS-isomorphic} if they are related by some version of the identity.
    \[ \xymatrix{ &C \ar[d] \ar@{=}[dl] \ar@{=}[dr] \\ C & F \ar[l] \ar[r] & C }
    \qquad
    \xymatrix{ &\hom_\Sigma(K,-) \ar[d] \ar[dl]_a \ar[dr]^b \\ C & F \ar[l] \ar[r] & C }
    \]
  \end{definition}
  \pause
  \begin{itemize}
  \item For univalent FOLDS-structures, need \alert<2>{the $\infty$-groupoid} $\mathsf{iso}(a,b)$ of FOLDS-isomorphisms.
  \item Need a version of the identity that contains \alert<2>{no more information} than an isomorphism $a\cong b$.
  \end{itemize}
\end{frame}

\begin{frame}
  \frametitle{FOLDS-isomorphisms}
  There is a canonical span with a common section relating $a$ to $b$:
  \[ C \xot{[\mathsf{id},a]} \;C \sqcup \hom_\Sigma(K,\blank)\; \xto{[\mathsf{id},b]} C \]
  But the legs may not be very surjective.
  \begin{definition}
    A \alert{FOLDS-isomorphism} from $a$ to $b$ is a version of the identity $C\ot F \to C$ such that
    \begin{enumerate}
    \item If $L\le K$, then $F_L$ is the above canonical span.
    \item Otherwise, $C_L \times_{M_L C} M_L F \otiso F_L \toiso C_L \times_{M_L C} M_L F$ are equivalences.
    \end{enumerate}
  \end{definition}
\end{frame}

\end{document}
